\chapter{General Hints on Optimization with \noun{EvA2}}

As you have come so far, you have hopefully learned a few things about
the concepts of \noun{EvA~2 }and got an idea on how optimization
in the framework can be done. Now we want to give a few more practical
hints on what to do with an unknown, probably difficult function.
Let's suppose you have a mathematical or algorithmic problem which
is infeasible to solve by means of mathematical analysis, gradients
are not computable in a simple way, and you haven't found a specialized
algorithm or heuristic in literature that solves it to the quality
you need, or you have found one but you don't understand it and can't
download it anywhere. All in all, you have a tough problem, \noun{MyToughProblem},
and you want to optimize it.

Let's also suppose you can implement \noun{MyToughProblem} in Java
(Sec.~\ref{sec:Quickly-Adding-Your-Problem}) or make it accessible
for\noun{ EvA~2} through an external interface (Sec.~\ref{sec:External-Interfaces}). 

You might now be tempted to ask: ``Exactly which optimizer do I need
to run with which parameters to solve \noun{MyToughProblem} once and
for all times?''.

And as you might have feared, this is a question which we cannot answer,
and maybe noone else can. But we will at least try to come a bit closer
to an answer here. There are several choices you have to make before
optimization, one of the earliest concern implementation: which representation
to use? E.g. in Sec.~\ref{sub:Accessing-Standard-Optimizers}, we
said a few things about double-valued and binary representation, and
many problems are natural to the one or the other implementation.
Combinatorical problems, for example, can be much easier projected
to a binary vector than to a double-valued one. Continuous functions
are a typical case for a real valued representation.
\begin{itemize}
\item which representation? 
\item how many optima? 
\item what range? 
\item will specialized operators help?
\item what about constraints?
\item what about multi objectives?
\end{itemize}
read the examples given in Sec.~\ref{sec:Using-the-API}, you might
So, for an unknown function

However, if you are dealing with an unknown function and you reckon
that it has quite a lot of optima, 
\begin{itemize}
\item Which options can be changed at all for optimizer X? -> anhang with
GOParameter classes? Some optimizers have no specialized GOParameters.
Access optimizer directly.
\end{itemize}