\chapter{Quickly Adding a Problem Class\label{sec:Quickly-Adding-Your-Problem}}

It is easy to integrate your own Java-coded problem in \noun{EvA2}
and optimize it using the GUI. Preconditions for the quick way are
only the two following ones:
\begin{itemize}
\item A potential solution to your problem can be coded as a double vector
or a BitSet.
\item You can implement the target function such that it produces a double
valued fitness vector from the double vector or BitSet.
\end{itemize}
If these simple conditions are met, you have the advantage of being
able to use the full ``Java-Power'' to implement a target function
and optimize it in \noun{EvA2}. Just follow these steps:
\begin{itemize}
\item Create an empty class (let's say, \texttt{ExampleProblem}) and assign
it to the package \texttt{eva2.problems.simple}. Put it in a directory called
``eva2/problems/simple/'' within your working directory.
\item Have the class inherit from \texttt{eva2.problems.simple.SimpleDoubleProblem}
or, depending on which datatype you want to use, from \texttt{eva2.problems.simple.Simple\-Binary\-Problem}.
Both base types can be used directly from the \noun{EvA2} jar-file
(which of course needs to be on the java classpath - in Eclipse, for
instance, add the \noun{EvA2} jar as ``External jar'' to the Java
build path through the project settings).
\item Implement the method \texttt{public int getProblemDimension() \{...\}}
whithin your ExampleProblem, which returns the number of dimensions
of the solution space, i.e. the length of the \emph{x} vector or size
of the \texttt{BitSet}, respectively. The problem dimension may be
a variable defined in your class, but it must not change during an
optimization run.
\item Implement the method \texttt{public double{[}{]}} \texttt{evaluate(double{[}{]}
x)} for double coded problems or \texttt{public} \texttt{double{[}{]}}
\texttt{evaluate(BitSet} \texttt{bs)} for binary coded problems, whithin
your ExampleProblem, where the fitness of a potential solution is
calculated.
\item Start the \noun{EvA2} GUI. \emph{Make sure that your working directory
is in the Java classpath}! From the problem list, select the \emph{SimpleProblemWrapper}
class as optimization problem. The wrapper class allows you to select
your ExampleProblem as target function. If you implemented a double
valued problem, you may also set a default range parameter, defining
the positive and negative bound of the solution space allowed. Now
select your prefered optimization method and start the optimization.
\end{itemize}