\chapter{Introduction}

\emph{\noun{EvA2}} (an \uline{Ev}olutionary \uline{A}lgorithms
framework, revised version \uline{2}) is a comprehensive heuristic
optimization framework with emphasis on Evolutionary Algorithms implemented
in Java%
\footnote{Oracle and \emph{Java} are registered trademarks of Oracle and/or its affiliates.%
}. It is a revised version of the \noun{JavaEvA} \cite{JOptDocumentation}
optimization toolbox, which has been developed as a resumption of
the former \noun{EvA} software package \cite{Wakunda97EvA}.

\noun{EvA2} integrates several derivation free optimization methods,
preferably population based, such as Evolution Strategies, Genetic
Algorithms, Differential Evolution, Particle Swarm Optimization, as
well as classical techniques such as multi-start Hill Climbing or
Simulated Annealing.

\noun{EvA2} aims at two groups of users. Firstly, the applying user
who does not know much about the theory of Evolutionary Algorithms,
but wants to use them to solve a specific application problem. Secondly,
the scientific user who wants to investigate the performance of different
optimization algorithms or wants to compare the effect of alternative
or specialized evolutionary or heuristic operators. The latter usually
knows more about evolutionary or heuristic optimization and is able
to extend \noun{EvA2} by adding specific optimization strategies or
solution representations. Explicit usage examples for \noun{EvA2 }are
given in \cite{Kron10EvA2}.

This document is, as the title says, not an extensive manual on the
\noun{EvA2} framework, but instead a short introduction hoping to
ease access to \noun{EvA2.} Thus, the document is mainly sketched
along use-cases and tries to deliver knowledge on a top-down basis,
with most important things first and details where required. Still:
as \noun{EvA}, just as mostly any larger software package, can become
tricky sometimes, it is not always possible to explain things without
cross-references. We hope that this document will, anyways, be a valuable
helper in working with \noun{EvA2}.

The document contains, of course, a Quick Start guide (Sec.~\ref{sec:Quick-Start})
also explaining the graphical user interface (GUI, Sec.~\ref{sub:Quickly-Using-GUI}).
Sec.~\ref{sec:External-Interfaces} contains hints on how to use
\noun{EvA2} with external programs, e.g. MATLAB%
\footnote{MATLAB is a registered trademark of The MathWorks, Inc. in the United
States and other countries.%
}. We provide a quick-and-simple way to add an application problem
implementation in Sec.~\ref{sec:Quickly-Adding-Your-Problem} and
describe more details of the API in Sec.~\ref{sec:Using-the-API}
to access further options and functionality. Finally, we propose some
literature sources for readings on Evolutionary and Heuristic Optimization
in Sec.~\ref{sec:Further-Reading} for the interested users.

\newpage{}